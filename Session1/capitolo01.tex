	\chapter{Il Giudizio}
	\setchaptertitle{1. IL GIUDIZIO}
	
	Un gruppo di persone \textit{(i giocatori)} si sveglia in un luogo misterioso. Si riesce a distinguere qualcosa data la poca luce che entra da quelle che sembrano pareti di legno. I movimenti di ciascuno sono limitati poiché tutti hanno le mani legate con delle corde \emph{(Possibile usare \textbf{Artista della Fuga}, \textbf{Animare corde}\dots)}. Guardando in giro ci si rende conto che il posto non è vuoto: ci sono dei vasi vuoti, dei sacchi di grano e una \textbf{cassa di legno da cui provengono dei suoni metallici}. Dall'esterno si sente rumore di zoccoli.\par Da ciò si potrebbe concludere di come il \textit{party} sia introppolato su un carro merci! Per di più, nessuno ricorda cosa sia accaduto precedentemente né se conosce o meno le persone di fronte a sé!\\
	\HRule{0.5pt}
	\emph{Da qui i giocatori possono inziare con le presentazioni, la descrizione di un loro outfit base e anche una breve storia sulle loro origini. Al termine del giro di presentazione la storia continua}.\\
	\HRule{0.5pt}
		Il carro sobbalza per il terreno accidentato e una parete della \textbf{cassa di legno} cade, rivelando delle armi. \emph{(Le armi della cassa saranno scelte dal GM. I giocatori possono tagliare le corde sfregandole contro le lame e possono decidere se equipaggiarsi o meno di queste)}.
 		 Si sente un rumore sordo e uno dei cavalli cadere.\par Il carro quindi si ferma all'improvviso; dopo pochi secondi si sentono dei versi grotteschi e un \textbf{nano}\footnote{Profilo a fine capitolo.}, sbloccata la porta sul retro, entra di fretta. \textit{"Non c'è tempo da perdere!"} dice mentre slega ognuno e prende per ciascuno un'arma da una cassa in fondo. \textit{"Me ne pentirò, ma ho bisogno di aiuto: sono arrivati!"}\dots\\
 		\HRule{0.5pt}
 		\textbf{Q\&A}
 		\begin{enumerate}
 			\item \textbf{Chi sta arrivando?} \textit{"La banda degli Spaccaossa. Un nome originale tanto quanto la loro ferocia"}.
 			\item \textbf{Chi fa parte della banda?} \textit{"Goblin! I più cattivi della zona!"}
 			\item \textbf{Chi sei?} \textit{"Sono Ulfgar, un povero contadino che, grazie a voi, diventerà ricco! Perciò non credete che dopo non vi rimetta a dormire, mi servite per riscattare la mia ricompensa"}.
 			\item \textbf{Di quale ricompensa parli?} \textit{"Davvero me lo stai chiedendo? Come se non lo sapessi\dots Non ti avrò picchiato troppo forte, mi auguro!"}
 		\end{enumerate}
 		\HRule{0.5pt}
 		Uscendo dal carro il \textit{party} viene circondato da quattro \textbf{Goblin}\footnote{Vedi Pathfinder, Bestiario.}. \textit{"Che stiamo aspettando? Attacchiamo!"}\dots\\
 		\HRule{0.5pt}
 		\emph{Inzia il combattimento. Per 4 Goblin il grado di sfida sarebbe di $1/3 + 4 = 12/3 = 4$, perfetto se i vostri PG partono dal livello 5 circa. In situazioni diverse toccherà al GM adattare. Se uno dei PG riceve un colpo fatale automaticamente verranno storditi tutti i membri del gruppo e verranno sottratti armi e parte degli oggetti, per cui distingueremo due casi:}\par
 		\begin{description}
 			\item[Vittoria] I Goblin vengono sconfitti e, improvvisamente, Ulfgar fa un balzo all'indietro, prende una boccetta dal suo marsupio e, coprendosi il volto, la lancia ai piedi del \textit{party}. Tutti svengono e il nano riporta ognuno di nuovo sul carro\dots
 			\item[Sconfitta] Dopo aver stordito tutti, i Goblin fuggono ridendo con i pochi averi delle loro vittime. Ulfgar poi, ripresosi dopo poco tempo, si occupa di riportare i suoi prigionieri sul carro\dots
 		\end{description}
 		\textit{"È ora di alzarsi, ragazzi!"} dice Ulfgar mentre attraversate un grande portone di legno. Il \textit{party} si trova, adesso, in una piccola cittadina; ci sono mura fatte di palizzate e pietre, qualche casa sparsa, una locanda e, al centro, una fortezza di pietra con alte mura e alte torri. Il carro si ferma\dots Una voce sconosciuta urla \textit{"Prigionieri, scendete!"}. Appena il \textit{party} scende dal carro, ecco che si accorge che il carro si è fermato sotto la fortezza e che un grande \textbf{Orco} dalla pelle grigio scuro, con una benda sporca sull'occhio sinistro\footnote{Profilo a fine capitolo.} era la fonte di quella voce misteriosa. L'orco, il capo di un banda di banditi, dà una grande borsa di monete a Ulfgar, stringendogli la mano. Dopodichè, il \textit{party} viene portato nei sotterranei della fortezza, in una grande cella comune. \textit{"Attenderete qui la vostra\dots esecuzione"} vi dice uno dei bandti. Si era capito ormai chi dava la caccia al \textit{party}, ma perchè? \textbf{Forse a questo poteva rispondere il loro capo}\footnote{Qui si può suggerire o meno ai giocatori di intraprendere un interrogatorio, presente nel capitolo successivo.}\dots\\
 		\HRule{0.5pt}
 		\emph{Da qui i giocatori possono esplorare liberamente l'ambiente. La porta della cella è facilmente scassinabile (A discrezione del GM decidere i risultati minimi delle prove). Esplorando, oltre a eventuali tesori posizionati dal GM, c'è una pergamena dietro una pietra che fuoriesce leggermente dal muro; dentro la pergamena c'è un testo in una lingua particolare}\footnote{Descrizione dell'oggetto a fine capitolo.}\\
 		\HRule{0.5pt}
 		Una volta usciti, il \textit{party} aveva avanti a sè due possibilità: \textbf{recarsi ai piani superiori} per affrontare il capo della banda o \textbf{cercare di fuggire} facendo attenzione a eventuali pericoli\dots\newpage
 		\section{Profili}
 			\textit{I seguenti profili sono descritti vagamente, la generazione di ulteriori caratteristiche è a discrezione del GM. Sono tuttavia disponibili delle schede già compilate (lingua inglese).}
 			\subsection{Ulfgar}
 				Ulfgar è un nano alto circa un metro e quaranta; ha sulla settantina d'anni.\par
 				È una creatura ben in forma: ha un fisico abbastanza massiccio, una folta barba legata con una cordicella e una folta chioma castana. Veste con una giacca di pelle sotto cui ha una maglia di ferro; indossa dei pantaloni di cotone verdi a cui, su un passante della cintura, è attaccato un marsupio di stoffa un po' malandato.\par
 				Possiede un grande martello \emph{(1d6 di danno)} e dei coltelli da lancio \emph{(1d4 di danno)}.\par
 			\subsection{Capo della banda degli orchi}
 				Kugdish, il capo degli orchi banditi, è una montagna di muscoli; è alto due metri e cinque, la sua età è sconosciuta. Ha un \textbf{Falcione} (\textit{2d8 + 9 danni}). Per le altre caratteristiche fare riferimento alla voce \textbf{Ogre} del \textit{\textbf{Bestiario}}.
 \newpage
 	\section{Quest secondarie}
 			\textit{Qui vengono descritti gli oggetti che possono far intraprendere avventure secondarie.}
 			\subsection{Pergamena delle celle}
 				Questa pergamena è un po' consumata e ingiallita. Se maneggiata da un personaggio di taglia grande o con un modificatore di  \textbf{Fortuna} tropp basso può necessitare di particolare attenzione nell'interazione con essa. La pergamena riporta scritto questo testo:\\\\
					\begin{center}
	 					{\LARGE {\Fontauri prova}}
	 				\end{center}