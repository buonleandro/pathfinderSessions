\chapter{Ambiente e dungeon}
\setchaptertitle{LUOGHI E DUNGEON}
	\emph{In questo capitolo introduttivo ci soffermeremo sulla struttura del mondo di gioco adottata per costruire questa sessione. Tuttavia, il GM più esperto può fare a meno di sfogliare queste pagine e usare le regole che più preferisce; infatti, questo vuole essere solo un capitolo riassuntivo e di supporto alle modalità di esplorazione e di interazione con l'ambiente di gioco. Regole sui tipi di terreno e sul clima non sono elencate.}\\
	
	\section{Regole ambientali}
	
		\emph{Ogni posto esplorabile ha delle proprietà, dal tipo di terreno al clima. In questa sezione le analizzeremo (verranno trattate solo le regole che riguardano il contenuto di questa avventura), includendo anche regole relative ai PG come fame e sete.}	
		
		\subsection{Acido}
			\emph{Chiunque sia a contatto con una sostanza acida riceve \textit{1d6} di danni ogni turno, a meno che non sia proprio immersa, in tal caso riceverà \textit{10d6} di danni ogni turno. Gli attacchi acidi valgono come turno di esposizione (\textit{1d6} di danni aggiuntivi).\\
			Gli acidi emanano fumi tossici, che valgono come veleno; se qualcuno li inala dovrà scegliere, per ogni turno di esposizione, se affrontare una prova di \textbf{Tempra} o se perdere un punto di \textbf{ Costituzione}. Tuttavia, il veleno scompare non appena ci si allontana dalla zona tossica. Una creatura in generale, se anche fosse immune all'acido, può anche \textbf{affogarci} (Vedi in \textbf{\nameref{wotah}}, sotto \textbf{Affogare}).}	
		
		\subsection{Acqua}\label{wotah}
			\emph{Qualsiasi personaggio può \textbf{guadare acque} relativamente calme senza affrontare prove. Allo stesso modo, \textbf{nuotare} in acque calme richiede solo una prova di \textbf{Nuotare} con CD di 10. I nuotatori addestrati possono subito \textbf{prendere 10}. Armature o equipaggiamento pesante rendono più difficile qualsiasi tentativo di nuotare.\\
			L'acqua agitata è molto più pericolosa. Un personaggi deve effettuare una prova di \textbf{Nuotare} con CD di 15 o una prova di \textbf{Forza} con CD di 15 per evitare di essere travolto. In caso di fallimento, il personaggio prende \textit{1d3} di danni non letale per turno (\textit{1d6} di danni letali se si incappa in rocce e cascate).\\
			L'acqua molto profonda non è solo nera come la pece, rappresenta anche un pericolo per la navigazione e, peggio, infligge danni da pressione di \textit{1d6} al minuto ogni 30 metri che un personaggio è sotto la superficie. Se un tiro salvezza di \textbf{Tempra} (CD di 15, +1 per ogni prova precedente) ha successo il nuotatore non subirà alcun danno in quel minuto. L'acqua molto fredda infligge \textit{1d6} danni non letali ogni minuto di esposizione.\\
			Per la caduta in acqua, riferirsi al punto \textbf{Caduta in acqua} della sottosezione \textbf{\nameref{cad}}.
				\subsubsection{Affogare}
					Un personaggio può trattenere il respiro per un numero di turni pari al doppio del suo punteggio di \textbf{Costituzione}. Se un personaggio spende un'azione standard o completa (lunga un turno), il tempo rimanente in cui il personaggio può trattenere il respiro viene ridotto di 1 turno. Dopo questo periodo di tempo, il personaggio deve affrontare, a ogni turno, una prova di \textbf{Costituzione} con CD di 10 per continuare a trattenere il respiro. A ogni round, la CD aumenta di 1. Quando un personaggio fallisce una prova di \textbf{Costituzione}, inizia ad annegare. Dopo un turno, cade in stato di incoscienza (0 pf). I personaggi incoscienti devono affrontare delle prove di \textbf{Costituzione} immediatamente. Una volta che viene fallita una di queste prove, perde 1 punto ferita. Nel turno successivo, annega.\\
					È possibile annegare in sostanze diverse dall'acqua, come sabbia, sabbie mobili, polvere fine, silos pieni di grano\dots}
		
		\subsection{Caduta}\label{cad}
			\emph{La caduta va divisa in categorie:
			\begin{description}
				\item[Caduta involontaria] Il soggetto cade accumulando \textit{1d6} di danni ogni 3 metri fino a un massimo di \textit{20d6}. È possibile lanciare incantesimi solo se si cade da almeno 150 metri o se l'incantesimo è immediato come \textbf{Caduta morbida}. Lanciare un incantesimo richiede una \textbf{prova di concentrazione} con CD pari a 20 + livello dell'incantesimo (risultato d20 + livello da incantatore + modficatore legato agli incantesimi; \textbf{Chierici, Druidi e Ranger}, come modificatore per gli incantesimi, usano il loro modificatore di \textbf{Volontà}; \textbf{Bardi, Maghi e Paladini} usano il loro modificatore di \textbf{Carisma}; infine, gli \textbf{Stregoni} usano il loro modificatore di \textbf{Intelligenza}). Anche se si lanciano incantesimi come \textbf{Teletrasporto} il momento della caduta rimane, quindi, anche se ci si sposta su una superficie piana, verrà sottratto il danno accumulato fino a quel momento.
				\item[Caduta volontaria\slash Acrobazia] Il soggetto, se cade perché ha saltato o se durante la caduta esegue \textbf{Acrobazia}, riceve \textit{1d6} di danni non letali (recuperabili con il riposo) prima degli \textit{1d6} di danni letali ogni 3 metri di caduta fino a un massimo di \textit{20d6}. La riduzione del danno, quindi l' aggiunta di \textit{1d6} di danni non letali prima dei \textit{d6} di danni letali, può essere cumulata su si cade su una superficie morbida o, per esempio, il fango.
				\item[Caduta in acqua] La caduta in acqua è un po'diversa: se lo specchio d'acqua in cui si cade è profondo almeno 3 metri, i primi 6 metri di caduta non fanno danno. I successivi 6 cumulano danno non letale (\textit{1d3} ogni 3 metri). Se si continua a cadere, viene cumulato danno letale (\textit{1d6} ogni 3 metri).\\
				Se un personaggio si \textbf{tuffa} in acqua dovrà superare una prova di \textbf{Acrobazia} o di \textbf{Nuotare} con CD pari a 15, questo finché l'acqua è profonda 3 metri per ogni 9 metri di caduta (se per esempio la caduta è di 18 metri, l'acqua dev'essere profonda 6 metri almeno). Tuttavia la CD aumenta di 5 ogni 15 metri di caduta in più.
				\item[Danno da oggetti in caduta] Benché non appartenga esattamente a questa sezione, vediamo ora i danni che provoca un oggetto se cade su un personaggio, basandoci su dimensione e altezza della caduta (questa regola può essere benissimo adattata dal GM):
				\begin{table}[h!]
					\begin{center}
						\label{tab:table1}
						\begin{tabular}{lccc} % <-- Alignments
							\textbf{Dimensione} & \textbf{Caduta} & \textbf{Caduta} & \textbf{Caduta}\\
							 & \textbf{meno di 9 m} & \textbf{tra 9 e 15 m} & \textbf{più di 15 m} \\
							\hline\\
							\textbf{Piccolo} & \textit{Dimezza} & \textit{2d6} & \textit{Raddoppia}\\
							\textbf{Medio} & \textit{Dimezza} & \textit{3d6} & \textit{Raddoppia}\\
							\textbf{Grande} & \textit{Dimezza} & \textit{4d6} & \textit{Raddoppia}\\
							\textbf{Enorme} & \textit{Dimezza} & \textit{6d6} & \textit{Raddoppia}\\
							\textbf{Gigantesco} & \textit{Dimezza} & \textit{8d6} & \textit{Raddoppia}\\
							\textbf{Colossale} & \textit{Dimezza} & \textit{10d6} & \textit{Raddoppia}\\
						\end{tabular}
					\end{center}
				\end{table}\\
			I \textit{Dimezza} e i \textit{Raddoppia} in tabella si riferiscono al risultato del lancio dei dadi. Un oggetto fragile arrecherà la metà del danno: se l'oggetto cade, per esempio, da 6 metri si tirano i dadi e si dimezza due volte il risultato, se cade da 16 metri non verrà raddoppiato il danno e così via; tuttavia, questo è a discrezione del GM. È possibile usare questa tabella anche per determinare il danno se viene lanciato un oggetto su un nemico: questo, se si accorge che l'oggetto sta cadendo su di lui o gli è stato lanciato, dovrà superare una prova di \textbf{Riflessi} (\textit{d20} più modificatori di \textbf{ Destrezza} e di \textbf{Riflessi}).
		\end{description}}
	
		\subsection{Caldo}
			\emph{Il caldo infligge danni non letali finché non ci si raffredda (si raggiunge l'ombra, ci si immerge in acqua, si sopravvive fino a notte fonda e così via\dots). Una volta che un personaggio ha accumulato tanti danni quanti sono i punti vita, i danni successivi saranno letali.\\
			Un personaggio in condizioni di caldo veramente forte (dai 33 gradi) deve affrontare una prova di \textbf{Tempra} ogni ora (CD di 15, +1 per ogni prova precedente) o incassare \textit{1d4} di danni non letali. Se si indossano vestiti o armature pesanti si avrà penalità di -4 alle prove. Un personaggio con l'abilità \textbf{Sopravvivenza} ottiene un bonus alla prova, mentre chi è incosciente riceve \textit{1d4} di danni non letali ogni ora.\\
			In caso di caldo elevato (dai 45 gradi), un personaggio dovrà affrontare la prova di \textbf{Tempra} o subire gli \textit{1d4} danni ogni 10 minuti. Se si indossano vestiti o armature pesanti si avrà penalità di -4 alle prove. Un personaggio con l'abilità \textbf{Sopravvivenza} ottiene un bonus alla prova, mentre chi è incosciente riceve \textit{1d4} di danni non letali ogni 10 minuti.\\
			Soffrendo il caldo, si rischia di avere un colpo di calore (\textbf{Affaticamento}), che si annulla col riposo in condizioni climatiche normali.\\
			Il caldo estremo (aria di 60 gradi o più, fuoco, acqua bollente, lava) infligge danni letali. Respirare l'aria di una tale temperatura infligge \textit{1d6} di danni ogni minuto, senza possibilità di salvataggio. In aggiunta, un personaggio dovrà affrontare la prova di \textbf{Tempra} o subire gli \textit{1d4} danni ogni 5 minuti. Se si indossano vestiti o armature pesanti si avrà penalità di -4 alle prove.\\
			L'acqua bollente arreca \textit{1d6} di danni al minuto; se si è immersi i danni sono di \textit{10d6} al minuto.}
				\subsubsection{Prendere fuoco}
					\emph{I personaggi esposti a incendi, olio bollente o fuochi magici non instantanei possono trovarsi capelli, vestiti ed equipaggiamento a fuoco. Per i fuochi magici è importante specificare se sono istantanei, questo perché se sono istantanei appaiono e spariscono immediatamente senza dare a fuoco niente.\\
					Chi prende fuoco deve superare una prova di \textbf{Riflessi} con CD 15. Ogni vota che viene fallita la prova si subiranno \textit{1d6} di danni e si potrà tentare la prova al turno successivo. Se si ha successo, il fuoco si spegnerà. Per estinguere il fuoco ci si può anche tuffare in acqua o avere un bonus di +4 alla prova se ci si rotola a terra o se si usa qualcosa come un mantello per spegnere le fiamme.\\	
					La prova va eseguita anche per ogni oggetto a fuoco.}
				\subsubsection{Diffusione del fuoco}
					\emph{Il fuoco fuori controllo si espande di un certo numero di caselle, a ogni turno. La presenza di oggetti infiammabili, vegetazione secca e vento possono contribuire a far diffondere le fiamme. Cospargere le vicinanze di materiale non infiammabile può contribuire a placare le fiamme, e ciò che è stato già bruciato non può esserlo ancora.\\
					Per verificare come si diffonde il fuoco lancia \textit{1d20} a ogni turno e fai riferimento alla seguente tabella:}\\
					\begin{table}[h!]
						\begin{center}
							\label{tab:table2}
							\begin{tabular}{cc}
								\textbf{Risultato d20} & \textbf{Reazione del fuoco}\\
								\hline\\
								\textit{1} & \textit{Il fuoco non cresce in questo turno}\\
								\textit{2} & \textit{Il fuoco cresce di 1 casella a nord}\\
								\textit{3} & \textit{Il fuoco cresce di 1 casella a est}\\
								\textit{4} & \textit{Il fuoco cresce di 1 casella a sud}\\
								\textit{5} & \textit{Il fuoco cresce di 1 casella a ovest}\\
								\textit{6} & \textit{Il fuoco cresce di 1 casella in ogni direzione}\\
								\textit{7-8} & \textit{Il fuoco non cresce in questo turno}\\
								\textit{9} & \textit{Il fuoco cresce di 2 casella a nord}\\
								\textit{10} & \textit{Il fuoco cresce di 2 casella a est}\\
								\textit{11} & \textit{Il fuoco cresce di 2 casella a sud}\\
								\textit{12} & \textit{Il fuoco cresce di 2 casella a ovest}\\
								\textit{13} & \textit{Il fuoco cresce di 2 casella in ogni direzione}\\
								\textit{14-18} & \textit{Il fuoco non cresce in questo turno}\\
								\textit{19} & \textit{Il fuoco cresce di 3 casella in ogni direzione}\\
								\textit{20} & \textit{Il fuoco cresce di 4 casella in ogni direzione}\\
							\end{tabular}
						\end{center}
					\end{table}
				\subsubsection{Incendi}
					\emph{Gli edifici che prendono fuoco sono inghiottiti dalle fiamme e rasi al suolo quasi immediatamente, a seconda della dimensione. Questi diventano insalvabili se nessun personaggio o NPC si accinge a estinguere le fiamme. Decidere quanto velocemente brucia un edificio è a discezione del GM: le \textbf{costruzioni piccole da un piano} bruciano in \textit{6d8} minuti; le \textbf{grandi abitazioni} (come le case delle città e le case dei mercanti) bruciano in \textit{4d20} minuti; \textbf{costruizioni più grandi della media} come ville, castelli e cattedrali bruciano in \textit{2d4} ore. Se un edificio è fatto completamente di materiali infiammabili brucerà in metà tempo, mentre edifici costruiti principalmente con materiali non infiammabili brucerà in metà del tempo in più. Alcuni edifici collasseranno mentre bruciano.}
				\subsubsection{Spegnere un fuoco}
					\emph{Per \textbf{spegnere un incendio} è necessario gettare sulle fiamme una grande quantità di acqua o altro materiale non infiammabile, come la terra. Una strategia efficace per l'estinzione rapida di un incendio consiste nel circondare l'area con materiale non infiammabile. Il giocatore dovrà, quindi, effettuare un attacco a distanza contro una \textbf{CA} di 10 per portare il materiale su una determinata casella. Di seguito sono elencati vari metodi per l'estinzione del fuoco:
						\begin{description}
							\item[Otre] Venti otri piene d'acqua estinguono una casella.
							\item[Secchio] Quattro secchi pieni di acqua o materiale non infiammabile estinguono una casella.
							\item[Serbatoio] 12 serbatoi di acqua estinguono una casella.
							\item[Calderone] Un calderone di acqua o materiale non infiammabile estinguono una casella.
							\item[Buco portatile] Un buco portatile riempito d'acqua o da altro materiale non infiammabile spegne un'area di 12 per 12 caselle.
							\item[Borsa conservante] Una borsa conservante di tipo I riempito con acqua o materiale non infiammabile estingue un'area di 3 per 3 caselle, il tipo II estingue un'area di 5 per 5, il tipo III estingue un'area di 7 per 7 e il tipo IV estingue un'area di 10 per 10.
						\end{description}
					}
				\subsubsection{Fumo}
					\emph{Un personaggio che inala fumo deve fare una prova di \textbf{Tempra} ogni turno (CD di 15, +1 per ogni prova precedente) o passare il turno a soffocare. Fallire la prova tante volte quanto il proprio modificatore di \textbf{Costituzione} farà cadere in uno stato di \textbf{Incoscienza}. Un personaggio che soffoca per 2 turni consecutivi prende \textit{1d6} di danni non letali.\\
					Il fumo oscura la vista, che risulta nell' \textbf{occultamento} (20\% di possibilità di mancare il bersaglio) dei personaggi al suo interno.}
				\subsubsection{Lava}
					\emph{
						La lava o il magma infliggono \textit{2d6} danni ogni turno di esposizione, tranne nel caso di immersione totale (come quando un personaggio cade nel cratere di un vulcano attivo), che infligge \textit{20d6} danni ogni turno.\\
						Il danno dalla lava (metà di quello da contatto, quindi \textit{1d6} o \textit{10d6}) continua \textit{1d3} turni dopo che non si è più esposti. L'immunità o resistenza al fuoco viene considerata come immunità o resistenza alla lava o al magma. Una creatura immune al fuoco potrebbe ancora affogare se completamente immersa nella lava.}
	
		\subsection{Fame e sete}
			\emph{I personaggi potrebbero trovarsi senza cibo o acqua e senza mezzi per ottenerli. Con clima normale, i personaggi medi hanno bisogno di almeno 1 litro di liquidi e circa 1 kg di cibo al giorno per evitare la \textbf{fame} (i personaggi piccoli ne richiedono la metà). In climi molto caldi, i personaggi hanno bisogno da due a tre volte più acqua per evitare la \textbf{disidratazione}.\\
			Un personaggio può fare a meno di \textbf{bere} per 1 giorno più un numero di ore pari al suo punteggio di \textbf{Costituzione}. Oltre questo, un personaggio deve effettuare una prova di \textbf{Costituzione} ogni ora (CD di 10, +1 per ogni prova precedente) o prendere \textit{1d6} di danni non letali. I personaggi che subiscono una quantità di danno non letale uguale ai loro punti ferita iniziano a subire danno letale.\\
			Un personaggio può stare \textbf{senza cibo} per 3 giorni, in crescente disagio. Oltre questo, un personaggio deve effettuare una prova di \textbf{Costituzione} ogni giorno (CD di 10, +1 per ogni prova precedente) o prendere \textit{1d6} di danni non letale. I personaggi che subiscono una quantità di danno non letale uguale ai loro punti ferita iniziano a subire danno letale.\\
			I personaggi che hanno subito danni non letali per mancanza di cibo o acqua sono \textbf{affaticati}. Il danno non letale causato dalla sete o dalla fame non può essere recuperato finché il personaggio non mangia o beve; nemmeno la magia guarisce questo tipo di danno.}
	
		\subsection{Freddo}
			\emph{Il freddo causa danni non letali a una creatura che ne soffre. Il freddo causa danni finché non ci si allontana dalla zona interessata dal fenomeno e non ci si riscalda. Se i danni non letali accumulati sono pari ai \textbf{Punti Vita} la vittima inizia a ricevere danni letali.\\
			I danni da freddo si hanno dai 4 gradi in giù; la vittima deciderà se ricevere gli \textit{1d6} danni o effettuare una prova di \textbf{Tempra} con CD 15 (+1 per ogni prova precedente) per ogni ora che ci si trova al freddo. Se chi fa la prova possiede anche l'abilità \textbf{Sopravvivenza} può applicare il bonus di quest'ultima al tiro della prova. Dai -18 gradi la scelta tra i danni o la prova va fatta ogni 10 minuti, a meno che non si abbia un outfit apposito. Dai -29 gradi la vittima subisce \textit{1d6} di danni senza possibilità di salvezza; in aggiunta, si deve scegliere tra la prova o \textit{1d4} di danni.\\
			Il freddo rigido provoca l'ipotermia (stato di \textbf{Affaticamento}), che può essere annullata grazie al riposo in un luogo almeno tiepido.}
				\subsubsection{Ghiaccio}
					\emph{Il movimento sul ghiaccio è diverso da quello su altri terreni, significa che per ogni casella di movimento sul ghiaccio bisogna muoversi di due. Le prove di \textbf{Acrobazia}, e in generale quelle legate a \textbf{Destrezza}, avranno una CD di +5. Chi resta abbastanza a contatto con il ghiaccio rischia di soffrire di un severo raffreddamento.}
	
		\subsection{Oscurità}
			\emph{I personaggi e le creature con \textbf{Scurovisione} sono capaci di vedere in condizioni di totale oscurità ma le creature con vista normale saranno considerate \textbf{cieche}. Lanterne e torce possono essere spente facilmente e creare oscurità, così come i fuochi magici e alcuni oggetti magici possono creare patine di oscurità impenetrabile.\\
			Una creatura accecata perde l'opportunità di eseguire un attacco di precisione e l'abilità di caricare. Per muoversi velocemente (la velocità normale si dimezza nell'oscurità), una creatura deve superare una prova di \textbf{Acrobazia} con CD di 10; se la prova non viene superata, si cade.\\
			Tutti i nemici di una creatura accecata risultano \textbf{occultati}, riducendo del 50\% la possibilità di essere colpiti in combattimento. Una creatura dovrebbe individuare il bersaglio per colpirlo, se è accecata; se non si individua il bersaglio, si colpire alla cieca, cioè l'attacco andrà in una direzione casuale. Per determinare la direzione di attacchi fisici o a distanza (inclusi \textbf{Incantesimi}) si tira un dado; verrà colpito il primo bersaglio in quella direzione.\\
			Se si è accecati, si perderà il modificatore di \textbf{Destrezza} alla \textbf{CA} e si subisce un malus di -2 alla \textbf{CA}. Inoltre, si avrà un -4 a ogni prova basata su \textbf{Destrezza} e \textbf{Forza} e tutte le abilità e le prove che fanno affidamento alla vista falliranno. Non si potranno, infine, usare \textbf{Attacchi basati sullo sguardo} e si sarà immuni a questi.\\
			Una creatura accecata può fare una prova di \textbf{Percezione} ogni turno come azione gratuita (CD uguale alla furtività dell'avversario). Una prova con successo permette sentire un avversario che non si riesce a vedere "da qualche parte". È quasi impossibile indicare con precisione, quindi, un avversario non visto. Una prova di \textbf{Percezione} che supera la CD di 20 rivela la posizione dell'avversario (ma questo è sempre \textbf{occultato} alla creatura accecata).\\
			Si può sempre andare a tentoni per cercare un nemico occultato; un personaggio può attaccare con le mani o con le armi nei quadrati adiacenti. Se una creatura si trova nelle caselle adiacenti, c'è il 50\% di possibilità di colpirla. Colpire in tal modo, tuttavia, non infligge danno, ma rivelerà l'attuale posizione del bersaglio; se questo si muove, la sua posizione è di nuovo sconosciuta. La posizione di una creatura non vista viene rilevata e puntata anche se questa colpisce un personaggio. Se però il nemico usa un attacco a distanza si conoscerà la direzione in cui si trova ma non l'esatta posizione.}
	
		\subsection{Soffocamento}
			\emph{Un personaggio senza \textbf{aria} può trattenere il respiro per 2 turni per ogni punto di \textbf{Costituzione}. Se un personaggio spende un'azione standard o completa (perdita del turno), il tempo rimanente per \textbf{trattenere il respiro} viene ridotto di 1 turno. Dopo questo periodo di tempo, un personaggio deve eseguire una prova di \textbf{Costituzione} con CD di 10 per continuare a trattenere il respiro. La prova deve essere ripetuta ogni turno, con la CD in aumento di +1 per ogni prova con successo precedente. Quando il personaggio fallisce, inizia a soffocare. Dopo un turno, cade incosciente (0 punti ferita). Nel turno successivo, scende a -1 punti ferita e inizia a morire. Dopo 3 turni, soffoca.}
			
	\section{Dungeon}
	\emph{I dungeon rappresentano uno dei punti cardine di un'avventura. Dedicate ai più coraggiosi, custodiscono segreti, tesori, mostri. Solo i più forti e intelligenti possono cavarsela in questi luoghi pericolosi.\\
		\begin{description}
			\item[Dungeon = Pericolo] Prima cosa importante, dungeon è una trappola mortale. Ogni architrave potrebbe nascondere una trappola, una grotta potrebbe essere casa di orchi crudeli, una tomba essere infestata da spiriti e una torre di uno stregone potrebbe essere cosparsa di rune mortali. Qualunque sia il dungeon il pericolo è sempre presente.
			\item[Dungeon = Storia] Gran parte dei dungeon sono rovine di luoghi caduti. Quando non sono covi di mostri e banditi costruiti appositamente per prendere alla sprovvista gli avventurieri, quindi, possono essere fonte di sapere. L'origine di un dungeon può anche far capire di più sulla sua struttura, aiutando così gli avventurieri che ci si vogliono addentrare.
			\item[Dungeon = Mostri] 
			I dungeon sono luoghi abbandonati, ma raramente disabitati. Vari mostri possono attaccarvi e tendervi imboscate; quel che è peggio è che gli abitanti dei dungeon conoscono il terreno su cui combattono.
		\end{description}
	ndipendentemente da vari motivi che possono spingere gli avventurieri a intraprendere la visita di un dungeon, questi saranno attratti maggiormente da un elemento: i dungeon nascondono \textbf{tesori}!}
	
	